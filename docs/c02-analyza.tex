\section{Analýza problému}

Základ zadání je v~zásadě triviální. Jedná se o~jednoduché procházení členů matice s~tím, že každý člen je porovnáván
se svými sousedními členy. Matici lze vyjádřit jako pole hodnot o~velikosti 1 bytu, neboť předpokládaná varianta formátu
PGM pracuje s~rozsahem hodnot odstínů šedi reprezentovaných čísly od 0 do 255.

\subsection[Okraj matice]{Okraj matice -- problém se sousedními členy}
První problém samozřejmě nastává v~případě, kdy se aktuální kontrolovaný člen nachází
\uv{na okraji} dané matice, neboť některé sousední členy se pak pomyslně nacházejí mimo tuto matici; tento problém je
však snadno eliminován tak, že jsou tito sousedé zcela ignorováni, nebo (což je řešení zvolené v~této práci) je jim
připsána zástupná hodnota, která nemá vliv na výsledek algoritmu (v~našem případě jde o~hodnotu, která představuje barvu
pozadí, tedy \verb|0x00|).

\subsection{Tabulka ekvivalence}
Druhým problémem je reprezentace tabulky ekvivalence\footnote{viz popis algoritmu CCL v~zadání}.

První možné řešení je jasné: lze ji vyjádřit polem, neboť v~použité variantě formátu PGM je maximální počet barev 256,
každou barvu lze tedy vyjádřit jedním bytem a v~poměru k~velikostem dnešních paměťových jednotek je 256 bytů
zanedbatelným nárokem. Po prvním pokusu o~implementaci je však problém s~touto reprezentací tabulky okamžitě zřejmý:
zadání sice předpokládá, že výsledný počet komponent (a tedy barev) v~obrázku nepřesáhne číslo 256, ovšem během prvního
průchodu může vzniknout barev mnohem více.

Je tedy nutné sáhnout po komplexnějším řešení. První změnu je potřeba provést v~reprezentaci matice, neboť 1 byte
na člen již není dostačující. Ve výsledné implementaci je tedy barva pixelu vyjádřena 4 byty. Po této změně již však
není praktické (a často ani možné) vyjádřit tabulku ekvivalence polem, neboť toto pole by mohlo mít až $ 2^{32} $ členů,
z~nichž každý je navíc 4 byty dlouhý. K~reprezentaci tabulky ekvivalence tedy byla zvolena jednoduchá mapa. Popis její
implementace lze nalézt v~následující sekci.
