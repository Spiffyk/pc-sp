\section{Zadání}

\textit{Tato sekce byla zkrácena. Plné znění zadání lze nalézt na webových stránkách předmětu KIV/PC.}

\vspace{0.5cm}

Naprogramujte v~ANSI C přenositelnou \textbf{konzolovou aplikaci}, která provede v~binárním digitálním obrázku
(tj. obsahuje jen černé a bílé body) \textbf{obarvení souvislých oblastí} pomocí algoritmu \textit{Connected Component
Labeling}\footnotemark[1] (dále \textit{CCL}) z~oblasti počítačového vidění.

Úkolem vašeho programu tedy je vytvořit výsledný soubor s~obarveným obrázkem v~uvedeném
umístění a s~uvedeným jménem. Vstupní i výstupní obrázek bude uložen v~souboru ve formátu
PGM\footnotemark[1].

Testujte, zda je vstupní obraz skutečně černobílý. Musí obsahovat pouze pixely s~hodnotou
\verb|0x00| a \verb|0xFF|. Pokud tomu tak není, vypište krátké chybové hlášení (anglicky) a oznamte chybu
operačnímu prostředí pomocí nenulového návratového kódu.

\footnotetext[1]{Algoritmus a formát PGM jsou popsány v~dokumentu se zadáním, viz poznámku na začátku sekce.}
