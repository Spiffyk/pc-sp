\section{Popis implementace}

Nejprve by bylo vhodné podotknout, že celý zdrojový kód je vcelku hojně okomentovaný a každá funkce je popsána
krátkým dokumentačním komentářem v~angličtině ve formátu velmi podobném JavaDocu známého z~vývojového prostředí
jazyka Java. Některá integrovaná vývojová prostředí tyto komentáře berou v~potaz a umějí je kontextově zobrazovat
v~dobře čitelné podobě. Ne všechny funkce jsou proto zmíněny či detailně popisovány v~této dokumentaci.

Lze říci, že celý program je rozdělen do čtyř celků. Prvním celkem je vstupní bod programu tvořený souborem
\verb|main.c|, který obsahuje funkci \verb|main()| a obstarává funkcionalitu ohledně spouštění jednotlivých procesů,
které má program vykonávat.

Druhým celkem je de-facto knihovna pro práci s~binární variantou souborů formátu PGM (\textit{Portable GrayMap}) tvořená
soubory \verb|pgm.c|, \verb|pgm_status_printer.c| a s~nimi souvisejícími hlavičkovými soubory.

Třetí celek obstarává zpracování obrázku pomocí algoritmu CCL a je tvořen soubory \verb|gmproc.c| a \verb|gmproc.h|.

Čtvrtý a poslední celek je jednoduchou implementací mapy, která je v~předchozím celku použita jako reprezentace tabulky
ekvivalence a rovněž je využita k~dodatečné úpravě obrázku po jeho zpracování (viz dále). Tento celek je tvořen soubory
\verb|simple_table.c| a \verb|simple_table.h|.


\subsection{Vstupní bod}

Tato část tvořená jediným zdrojovým souborem \verb|main.c| obsahuje vstupní bod programu (funkci \verb|main()|), který
obstarává spouštění dalších procesů.

Nejprve je zkontrolováno, zda je program správně spouštěn pomocí příkazu
\verb|ccl.exe <vstupni_soubor> <vystupni_soubor>|. Pokud tomu tak není, je tato skutečnost oznámena uživateli a program
končí se stavovou hodnotou 1 (konstanta \verb|EXIT_MISSING_ARGS|).

Dále program doplní do cest ke vstupnímu a výstupnímu souboru příponu \verb|.pgm|, pokud přípony zcela chybí. Toto je
provedeno funkcí \verb|add_extension()|. V~případě, že přípony chybí, by ve velmi výjimečných případech mohlo dojít
k~chybě s~alokací paměti, neboť je cesta k~souboru s~doplněnou příponou zapsána na novou, dynamicky přidělenou adresu.
I~přes nízkou pravděpodobnost této události je tato chyba ošetřena a pokud k~ní dojde, skončí program se stavovou
hodnotou 6 (konstanta \verb|EXIT_ALLOC_ERROR|).

Po získání cest ke vstupnímu a výstupnímu souboru je pomocí níže popsané knihovny pro čtení PGM souborů přečten vstupní
soubor a je uchován v~paměti k~dalšímu zpracování. Během čtení může dojít k~několika různým chybám, které knihovna hlasí
pomocí svých stavových hodnot (k~nim více v~následující sekci). Pokud k~některé z~chyb dojde, program vypíše informaci
o~stavové hodnotě knihovny v~uživatelsky čitelné podobě a ukončí se s~vlastní stavovou hodnotou 2 (konstanta
\verb|EXIT_READ_ERROR|).

Po načtení vstupního souboru je pomocí stejné knihovny naalokován a zinicializován výstupní obrázek. Při alokaci může
opět dojít k~chybám, které knihovna opět hlásí pomocí svých stavových hodnot. Program opět v~případě chyby vypíše
informaci o~této skutečnosti a ukončí se se stavovou hodnotou 3 (konstanta \verb|EXIT_CREATION_ERROR|).

Nyní je vše připraveno a program volá funkci modulu ke zpracování vstupního obrázku pomocí CCL a zápisu do výstupního
obrázku. Tato funkce má stavovou návratovou hodnotu, pomocí které je nahlášen výsledek procesu. Pokud se proces zdaří,
je tato návratová hodnota nulová a program pokračuje v~činnosti. Pokud dojde při zpracování k~chybě, vypíše se této
chyby a do konzole a program končí svou činnost se stavovou hodnotou 4 (konstanta \verb|EXIT_PROCESS_ERROR|).

Po zdařilém pokusu o~zpracování obrázku již zbývá jen zápis výsledného obrázku na disk, který je opět proveden pomocí
níže popisované knihovny a opět může hlásit chyby, které jsou případně vypsány a program v~jejich případě končí se
stavovou hodnotou 5 (konstanta \verb|EXIT_WRITE_ERROR|).

Nakonec se již jen uvolní paměť, která byla ve funkci \verb|main()| dynamicky alokována a program končí se stavovou
hodnotou 0 značící úspěch (standardní konstanta \verb|EXIT_SUCCESS|).


\subsection[Knihovna pro práci s~PGM]{Knihovna pro práci s~binárními soubory typu PGM}

Tato knihovna obsahuje funkce a strukturu pro práci se soubory typu PGM. Struktura \verb|greymap| obsahuje informaci
o~rozměrech obrázku a jednorozměrné pole 32-bitových neznaménkových celých čísel představujících barvy pixelů obrázku.
Celá knihovna je rozdělena do pěti souborů, jejichž názvy začínají \verb|pgm|, k~jejímu použití se ovšem používá pouze
hlavičkový soubor \verb|pgm.h|, který zajistí vše potřebné. Výkonný kód všech níže uvedených funkcí knihovny se až
na výjimky, které budou uvedeny, nachází ve zdrojovém souboru \verb|pgm.c|.

Pomocí funkce \verb|greymap_read()| lze soubory PGM číst. Funkce má navíc zavedenu funkcionalitu pro ignorování řádek
uvozených znakem \verb|#|, které jsou ve formátu PGM definovány jako komentáře. Takové komentáře do souborů vkládá
například program \textit{GNU Image Manipulation Program} (zkráceně \textit{GIMP}), který byl využit k~tvorbě vlastních
testovacích obrázků při vývoji. Tato funkce nastavuje stavovou hodnotu knihovny dle svého výsledku, více níže.

Funkce \verb|greymap_write()| zapíše obrázek vyjádřený strukturou \verb|greymap| na disk. Opět dle své úspěšnosti
nastaví stavovou hodnotu.

Knihovna dále obsahuje funkci pro alokaci nového obrázku \verb|greymap_create()|. Ta alokuje strukturu \verb|greymap|
a její pole pixelů v~paměti a zinicializuje její informace o~rozměrech na zadané hodnoty. Pole pixelů
není na počátku nijak inicializováno; je nutné jej inicializovat ručně. Tato funkce rovněž nastavuje stavovou hodnotu
knihovny podle úspěšnosti.

Další potřebnou funkcí je \verb|greymap_free()|, která se stará o~uvolnění části paměti, kde se nachází obrázek.
Funkce sama nastavuje ukazatel na strukturu \verb|greymap| na hodnotu \verb|NULL|. Funkce \verb|greymap_free()| vždy
nastavuje stavovou hodnotu knihovny na 0 značící úspěch.

Další dvě funkce pro práci se strukturou \verb|greymap| jsou \verb|greymap_get_pixel()| a \verb|greymap_set_pixel()|.
V~tomto pořadí slouží k~získávání a zápisu hodnot pixelů dle zadaných \textit{souřadnic}.

Většina výše uvedených funkcí nastavuje stavovou hodnotu dle svého výsledku. Tato stavová hodnota je ukládána v~proměnné
\verb|status_code|, která je viditelná pouze uvnitř knihovny. Pro získání stavové hodnoty vně knihovny slouží funkce
\verb|greymap_status()|, která tuto hodnotu vrací. Veškeré konstanty stavových hodnot včetně jejich krátkých anglických
popisků se nacházejí v~souboru \verb|pgm_status.h|.

Poslední funkcí knihovny je \verb|greymap_status_fprint()|, ke které je také přidruženo makro
\verb|greymap_status_print()|. Funkce vypíše popis stavu v~uživatelsky čitelné podobě do zadaného proudu. Makro pak volá
tuto funkci tak, aby popis vypsala do standardního výstupu (tedy konzole). Výkonný kód funkce se nachází ve zdrojovém
souboru \verb|pgm_status_printer.c|.


\subsection[Zpracování obrázku pomocí CCL]{Zpracování obrázku pomocí algoritmu CCL}

Veškerá funkcionalita související s~algoritmem CCL je implementována ve zdrojovém souboru nazvaném \verb|gmproc.c|,
v~němž se nachází jediná z~vnějšku viditelná funkce \verb|process_greymap()|.

Tato funkce nejprve zkontroluje, že rozměry vstupního i výstupního obrázku programu jsou totožné a pokud nejsou,
skončí chybou.

Dále je výstupní obrázek inicializován tak, aby všechny jeho pixely měly hodnotu \verb|0x00| a rovněž je alokována mapa
(pro implementaci mapy viz sekci níže), která slouží jako tabulka ekvivalence.

Po těchto krocích následuje první průchod algoritmu CCL. Během tohoto průchodu jsou pro každý pixel získávány
do osmičlenného pole \verb|neighbors| získávány hodnoty sousedních pixelů pomocí funkce \verb|get_neighbors()|.
Implementace funkce \verb|greymap_get_pixel()| (viz výše) zajišťuje, že sousední pixely, jejichž souřadnice se nacházejí
mimo obrázek, jsou zastoupeny hodnotou \verb|0x00| a tudíž mají v~algoritmu význam pozadí, nikoliv bílé plochy.

Pro každý pixel, mezi jehož sousedními pixely se nachází alespoň dva takové, které mají vzájemně odlišnou barvu a nemají
barvu pozadí, je vytvořen záznam v~tabulce ekvivalence, kde je barva sousedního pixelu použita jako klíč. Pokud již klíč
v~tabulce existuje, zjišťuje se, zda nelze jeho odpovídající hodnotu v~tabulce již zapsanou použít jako další klíč (tedy
zda se další hodnota nerovná klíči).

Zbytek prvního průběhu probíhá přesně, jak je popsáno v~zadání, přičemž veškeré změny jsou zapisovány do výstupního
obrázku.

Během druhého průchodu probíhá aplikace tabulky ekvivalence, tedy všechny barvy, které jsou vzájemně ekvivalentní,
jsou převáděny na jednu společnou barvu, určenou tabulkou ekvivalence.

Dále je použita funkce \verb|fit_colors_to_char()|, neboť se může stát, že hodnoty pixelů přesahují maximální hodnotu,
kterou je možné vyjádřit jedním bytem. Tato funkce pomocí další vlastní mapy zajistí, že jsou všechny barvy obsažené
v~obrázku v~rozsahu 0 až 255.

Posledním krokem je použití funkce \verb|make_colors_distinguishable()|, která upraví obrázek tak, aby rozdíl mezi
jednotlivými barvami byl větší než 1 (kvůli rozpoznání správného výsledku lidským okem), pokud je to možné.


\subsection{Implementace mapy}

Program obsahuje jednoduchou implementaci mapy, jejíž výkonný kód se nachází v~souboru \verb|simple_table.c|.
V~hlavičkovém souboru jsou definovány struktury, z~nichž jedna představuje mapu a druhá záznam.

Mapa je realizována pomocí pole s~nastavitelnou délkou. Každý člen pole je záznam, obsahuje ukazatel
na případný další záznam, čímž vzniká datová struktura spojového seznamu. Pole je jakousi hashovací tabulkou. Hash kód
záznamu je určen tak, že je spočítán zbytek po celočíselném dělení jeho \textit{klíče} nastavenou \textit{délkou pole}.

Mapu lze alokovat funkcí \verb|simple_table_create()|, které lze jako argument předat velikost hashovací tabulky, nebo
makrem \verb|simple_table_default()|, které nastaví její velikost na 255. Velikost tabulky může být jakákoliv a má
především vliv na výkon a paměťové nároky mapy.

Hodnoty do mapy lze přidávat funkcí \verb|simple_table_put()| a číst funkcí \verb|simple_table_get()|. Funkce
pro odebírání záznamů nebyla implementována, neboť je pro účely této semestrální práce nepotřebná.

Paměť použitá mapou se uvolňuje pomocí funkce \verb|simple_table_free()|, která rekurzivně uvolní i paměť užívanou
veškerými záznamy v~ní obsaženými.
